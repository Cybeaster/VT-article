\documentclass[12pt,a4paper]{article}
\usepackage[utf8]{inputenc}
\usepackage[russian]{babel}
\usepackage[OT1]{fontenc}
\usepackage{amsmath}
\usepackage{amsfonts}
\usepackage{amssymb}
\usepackage{hyperref}
\usepackage[left= 3 cm, right = 3 cm, top = 3.0 cm, bottom = 3.0 cm]{geometry}
\begin{document}
\begin{flushleft}

\begin{center}
\section*{\textit{Введение}}
\end{center}
Целью данной работы является ознакомление с кафедрой \textbf{Вычислительной Техники} и разбор архитектуры операционной системы \textit{Linux}. Вследствие этого были сформулированы следующие задачи:
\begin{itemize}
\item Ознакомиться с кафедрой \textbf{ВТ}
\item Изучить \textit{Иерархическую структуру \textit{Linux}}
\item Детально рассмотреть \textit{Кёрнел Linux}
\end{itemize}
\newpage
\tableofcontents
\newpage
\section{\textit{Описание кафедры Вычислительной Техники}}
\subsection{Современное состояние материально-технической базы}
\begin{flushleft}
\large
На данный момент кафедра располагает более 70 ПК, имеет свой кафедральный сервер с автономным энергообеспечением для защиты имеющейся информации. Кроме этого присутствует 1-й сегментная суперкомпьютерная грид-система с включенных в нее терминальных классов \emph{<<Тонкий клиент>>}. \textbf{(ауд. 304, 307, 320)}
\\
В каждой аудитории есть свой Мультимедийный проектор, как стационарный, так и переносной. А в аудиториях \textbf{303 и 318} имеются интерактивные доски. В дополнение к существующему оборудованию для занятий по дополнительному образованию \emph{(подготовка «Web-дизайнеров»)} кафедрой закуплено и уже активно используется специальное художе-ственное оборудование в  \textbf{ауд. 313}. Всего в настоящее время установле-но оборудования на сумму, превышающую 21 млн. рублей. В будущем планируется расширение на базе \textit{<<Grid System>>} и \textit{<<Cisco Systems>>} что даст возможность сильно улучшить качество преподавания и расширить возможности научно-исследовательских работ.
\\
Приказом №162 от 11 июня 2010 г. на кафедре \textbf{ВТ} образован научный центр \textit{<<Прикладной Анализ Данных>>}. Его основными задачами являются: оказание различных научно-исследовательских и образовательных услуг, не предусмотренных университетскими планами научных работ и основными учебными планами направлений и специальностей, по которым ведется обучение в НГТУ. За все время работы центра было подготовлено свыше 56 специалистов в области \textit{<<Web-дизайн>>} и в области \textit{<<Сетевых специалистов Cisco Systems>>}.
\\
По итогам выполнения ИОП \textit{<<Высокие технологии>>} приказом по университету №205 от 2 марта 2009 г. на базе средств \textit{«Grid System»} был создан центр коллективного пользования \textit{«Распределенные вычислительные системы>>}.Основными направлениями его деятельности яв-ляются: предоставление доступа сотрудников, студентов и сторонних лиц к высокопроизводительным вычислительным ресурсам, развитие совместно с другими подразделения университета информационно-вычислительной среды и связь ее с глобальными грид-сетями, обучение и переобучение студентов и специалистов суперкомпьютерной технике, распределенным и параллельным информационным технологиям, про-ведение НИР в этом направлении.
\\
Приказом №2129 от 27 декабря 2011 г. в рамках инновационного гранта на кафедре был создан \textit{«Информационно-аналитический центр информационных технологий и цифровых систем»}, основными задачами которого являются: обеспечение инновационных подразделений НГТУ и других внутренних и внешних потребителей профильной аналитической научно-технической информацией, касающейся мониторинга и анализа текущего состояния дел по тематике центра, прогноз ее развития, экспертиза инноваций в отрасли информационных технологий и цифровых систем, оказание помощи в участии в профильных платформах и образовательных услуг.
\end{flushleft}
\subsection{Организация учебного процесса на кафедре}
\begin{flushleft}
\large
Богатая материально-техническая база кафедры позволяет обеспечить студентов возможностями обучения разнородным средствам вычислительной техники и телекоммуникаций. Так, например, установленный сегмент грид-системы позволяет студентам обучаться и работать на типовых персональ-ных компьютерах, рабочих станциях, как отдельных элементах, так и встро-енных в грид-систему
\\
Кафедра \textbf{ВТ} ведет подготовку по специальностям, предаставленным в таблице.
\newpage
\begin{center}
\textbf{\Large Бакалавр}
\end{center}
\begin{tabular}{|| c || c ||}
\hline
\emph{Направление специальности} & \emph{Код} \\ \hline
\textit{Информатика и вычислительная техинка \textbf{(Очная форма)}} & \textbf{09.03.01} \\
\textit{Информатика и вычислительная техника \textbf{(Заочная форма)}} & \textbf{09.03.01} \\
\textit{Программная инженерия \textbf{(Очная форма)}} & \textbf{09.03.04 } \\
\hline
\end{tabular}

\begin{center}
\textbf{\Large Магистратура}
\end{center}

\begin{tabular}{|| c || c ||}
\hline
\emph{Направление специальности} & \emph{Код} \\ \hline
\textit{Информатика и вычислительная техника \textbf{(Очная форма)}} & \textbf{09.04.01} \\
\textit{Информатика и вычислительная техника \textbf{(Заочная форма)}} & \textbf{09.04.01} \\
\textit{Программная инженерия \textbf{(Очная форма)}} & \textbf{09.04.04} \\
\hline
\end{tabular}

\begin{center}
\textbf{\Large Аспирантура}
\end{center}

\begin{tabular}{|| c || c ||}
\hline
\emph{Направление специальности} & \emph{Код} \\ \hline
\textit{Управление в технических системах} & \textbf{27.06.01} \\
\textit{Информатика и вычислительная техника} & \textbf{09.06.01}\\
\hline
\end{tabular}
\\
В рамках образовательной работы кафедра осуществляет следующую деятельность:
\\
\begin{itemize}
\item Обеспечение лабораторных работ, практических занятий, семи-наров, лекций, презентаций по проблемам сетевых технологий и защите информации в рамках образовательных программ. В настоящее время существует 252 учебных курса.
\item Адаптация учебных и сертификационных материалов \textit{Cisco} для образовательных целей.
\item Разработка учебно-методических комплексов для дисциплин, по которым в лабораториях будут проводиться занятия. В связи с появлением в университете двухуровневой системы подготовки выпускников (бакалавр - магистр), кафедра проводит большую работу по разработке соответствующих учебных планов, рабо-чих программ и учебно-методических комплексов по всем трем уровням обучения: бакалавриат, специалитет, магистратура.
\item Создание филиала академии \textit{Cisco Systems} с целью подготовки сертифицированных специалистов в соответствии с требованиями уровня \textit{CCNA}.
\item Организация академии \textit{Microsoft} и создание международного центра сертификации \textit{VUE}.
\item Подготовка кадров высшей квалификации. Центр \textit{«Прикладной анализ данных»}, созданный при кафедре в 2010 г., осуществляет повышение квалификации по программам: \textit{«Компьютерная гра-фика} и \textit{<<Веб-дизайн»} (240 часов), \textit{«Сетевой специалист>>}, \textit{<<Сертифицированный Cisco»} (72 часа), \textit{«Средства и технологии презентации в образовательном процессе»} (72 часа).
\item Переподготовка кадров. В 2003 г. на кафедре была открыта подготовка по 2-годничной программе дополнительного к выс-шему образованию \textit{«Специалист в области компьютерной графики и веб-дизайна»}, а с 2011 г. также \textit{«Системный инженер(специалист по эксплуатации аппаратно-программных комплексов, персональных ЭВМ и сетей на их основе)»} (срок обучения 1,5 года), \textit{«Разработчик профессионально ориентированных компьютерных технологий»} (2,5 года),\textit{«Преподаватель информатики»}  (1,5 года).
\end{itemize}
\end{flushleft}
\subsection{Организация научно-исследовательской работы на кафедре}
\begin{flushleft}
\large
\textbf{НИР} студентов подразделяются на:\\
\begin{itemize}
\item Учебно-исследовательскую работу студентов – работу, включаемую в учебный процесс
\item Собственно \textbf{НИРС} – работу, выполняемую во внеучебное время.
\end{itemize}
Научно-исследовательская работа студентов, включаемая в учебный процесс, осуществляется кафедрой в следующих формах:
\begin{itemize}
\item Выполнение лабораторных работ, домашних работ, курсовых и дипломных работ, содержащих элементы научных исследований
\item Введение элементов научного поиска в практические и семинарские занятия
\item Выполнение конкретных нетиповых заданий научно-исследовательского характера в период производственной и преддипломной практик
\item Ознакомление с теоретическими основами методики, постановки, организации и выполнения научных исследований, планирования и проведения научного эксперимента и обработки полученных данных
\item Участие в работе студенческих научных семинаров
\end{itemize}
Учебная научно-исследовательская работа студентов \textbf{(УИРС)} является одним из важнейших средств повышения качества подготовки и воспитания специалистов с высшим образованием, обладающих навыками исследования и способных творчески применять в практической деятельности.
\\
Научно-исследовательская работа студентов \textbf{(НИРС)}, организуемая во внеучебное время, включает следующие формы:\\
\begin{itemize}
\item Участие в работе студенческих научных коллективов
\item Участие в работе проблемных научных групп на профилирующих (выпускающих) кафедрах
\item Участие в выполнении хоздоговорной тематики кафедры.
\end{itemize}
Формы и методы \textbf{НИРС} зависят от уровня подготовки студентов. На младших курсах преобладают такие формы НИРС как написание рефератов, выполнение расчетных работ, перевод литературы и др. На старших курсах – реальное курсовое и дипломное проектирование, постановка и модернизация лабораторных работ, участие студентов в подготовке и проведении научных экспериментов, выполнение хоздоговорных научно-исследовательских работ.\\
В 1969г. на кафедре сложилось научное направление по исследованию и построению распределенных вычислительных систем (научный руководитель - к.т.н., доцент В.И.Жиратков), а в 1974г. - направление по разработке специализированных вычислительных устройств для автоматизации научных исследований (научный руководитель - к.т.н., доцент В.И. Соболев).
\\
В данный момент ведутся исследования по следующим направлениям:
\\
\begin{tabular}{||p{0.8\linewidth} || c||c||}
\hline
\emph{Название} & \textbf{Руководитель}\\
\hline
\textit{Алгоритмы, методы создания автоматизированных систем сбора и
обработки данных} & \textbf{Вихман В.В.}\\
\hline
\textit{Гибридные системы искусственного интеллекта} & \textbf{Гаврилов А.В.} \\
\hline

\textit{Глубокие и импульсные нейронные сети} & \textbf{Гаврилов А.В.} \\
\hline

\textit{Когнитивная робототехника} & \textbf{Гаврилов А.В.} \\
\hline

\textit{Многопользовальтельские операционные системы. Разработка и
оценка (анализ) алгоритмов управления вычислительными функциями. Моделирование функций управления} & \textbf{Коршикова Л.А.} \\
\hline

\textit{Применение теории нелинейных динамических систем в медицине} & \textbf{Рабинович Е.В.} \\
\hline

\textit{Разработка и исследование быстродействующих алгоритмов и
средств цифровой обработки сигналов} & \textbf{Овчеренко В.А} \\
\hline

\textit{Разработка методов и средств визуализации диагностических обра-
зов и исследование особенностей их восприятия человеком} & \textbf{Яковина И.Н} \\
\hline

\textit{Разработка методов и средств описания поведения климата с ис-
пользованием нечеткой логики} & \textbf{Яковина И.Н.} \\
\hline
\textit{Разработка моделетики для систем медицинской диагностики} & \textbf{Яковина И.Н} \\
\hline
\textit{Разработка модели нейросетевой сегментации изображений} & \textbf{Кугаевских А.В.} \\
\hline
\textit{Разработка новой технологии диагностики по биологически активным точкам человека} & \textbf{Яковина И.Н.} \\
\hline
\textit{Системный анализ, управление и обработка информации} & \textbf{Казанская О.В.} \\
\hline
\textit{Статистическая обработка данных} & \textbf{Бычков М.И.} \\
\hline
\textit{Технология сейсмических измерений: сейсморазведка на отраженных волнах, гидроразрыв пласта} & \textbf{Рабинович Е.В.} \\
\hline
\textit{Цифровая обработка сигналов: повышение скорости обработки данных, сжатие данных} & \textbf{Рабинович Е.В.} \\
\hline
\end{tabular}
В рамках научно-
исследовательской работы кафедра обеспечивает следующие направления научно-исследовательской и производственной деятельности:
\begin{itemize}
\item Исследование и разработка современных систем и средств для построения глобальных информационно-коммуникационных сетей.
\item Теоретическое и экспериментальное исследование принципов работы и организации глобальных сетей, алгоритмов, протоколов, аппаратно-программных реализаций и систем сетевых технологий.
\item Разработка, создание и исследование сетевого программного обеспечения и алгоритмов.
\item Выбор и применение средств защиты информации в корпоративных информационных системах и сетях.
\item Исследование и разработка методов и средств проектирования архитектуры аппаратно-программных сетевых комплексов и их компонентов
\end{itemize}
За период с 2001 г. по 2011 г. штатными сотрудниками и совместителями было опубликовано более 760 научных работ, получено 7 патентов, 9 свидетельств о регистрации программ для \textit{ЭВМ} и 1 – баз данных. Кроме того, 199 работ опубликовали аспиранты и студенты без соавторства с руководителями.
\\
Ежегодно представители кафедры участвуют в организации и проведении не менее пяти различных отечественных и международных научных мероприятий в качестве председателей и/или членов оргкомитетов, руководителей секций. Все профессора кафедры ежегодно являются членами диссертационных советов в НГТУ, НГУ, СибГУТИ, других НИИ и вузов.
\\
На кафедре проводятся школы: собирается группа учёных или коллектив исследователей, выполняющих в долгосрочном периоде под руководством лидера определенную научно-исследовательскую программу, решающую четко сформулированную научную задачу или комплекс задач. В них принимают участие известные ученые, академики РАН, доктора наук из разных НИИ, вузов и городов. Итогом проведения данных школ является ознакомление магистрантов и аспирантов кафедр ВТ и СИТ, а также других кафедр факультета, других вузов из России и зарубежья с новейшими разработками, выполняемыми в академических институтах России За последние 5 лет в рамках работы научной школы, получено 3 патента, издано 2 монографии, опубликовано 33 статьи в реферируемых журналах. В настоящее время работает научная школа прикладного многофункционального статистического анализа сигналов и данных
\end{flushleft}


\section{\textit{Архитектура Linux}}
\subsection{}
\subsection{}

\end{flushleft}

\end{document}